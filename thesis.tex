\documentclass[master=mai,masteroption=bda,oneside]{kulemt}
\setup{% Remove the "%" on the next line when using UTF-8 character encoding
	%inputenc=utf8,
	title={Explaining time series clusters},
	author={Simiao Lin},
	promotor={Prof. dr. ir. H. Blockeel\and Dr. ir. W. Meert},
	assessor={Oscar Mauricio Agudelo},
	assistant={Dr. A. Yurtman}}
% Remove the "%" on the next line for generating the cover page
%\setup{coverpageonly}
% Remove the "%" before the next "\setup" to generate only the first pages
% (e.g., if you are a Word user).
%\setup{frontpagesonly}

% Choose the main text font (e.g., Latin Modern)
\setup{font=lm}

% If you want to include other LaTeX packages, do it here. 

% Finally the hyperref package is used for pdf files.
% This can be commented out for printed versions.
\usepackage[pdfusetitle,colorlinks,plainpages=false]{hyperref}
\usepackage{algorithm}
\usepackage{algorithmic}
\usepackage{multirow}
\usepackage{amsmath} 
\usepackage[capitalise]{cleveref}
%\usepackage{flafter}
\usepackage{placeins}

\DeclareMathOperator*{\argmax}{argmax} % thin space, limits underneath in displays

%%%%%%%
% The lipsum package is used to generate random text.
% You never need this in a real master's thesis text!
\IfFileExists{lipsum.sty}%
{\usepackage{lipsum}\setlipsumdefault{11-13}}%
{\newcommand{\lipsum}[1][11-13]{\par And some text: lipsum ##1.\par}}
%%%%%%%

%\includeonly{chap-n}
\begin{document}
	
	\begin{preface}
		Throughout the writing of this thesis, I have received a great deal of support and assistance.
		
		I would first like to thank my advisor, Aras Yurtman. You are always so supportive and patient. I learn a lot about how to research you at our weekly meeting. I could not have completed this thesis without the advice and encouragement you give me all the time. 
		
		I would also like to acknowledge my supervisors, Wannes Meert and Prof. Hendrik Blockeel,  whose expertise was invaluable in formulating the research questions and methodology. 
		
		In addition, I would like to thank my accessors for reading the text.
		
		Finally, my gratitude also goes to my family for their sympathetic ear. You are always there resting my mind outside the research.
		
		
	\end{preface}
	
	\tableofcontents*
	
	\begin{abstract}
		%An important aspect of applying machine learning models to real-world use cases is interpretability. These novel features need to be explained or shown to domain experts. In this thesis we make use of the DTW (Dynamic Time Warping) distance measure, DBA (DTW barycenter averaging), and constraints from $ \mathrm{COBRAS^{TS}} $ to summarize why a set of time series are clustered together and how one cluster differs from other clusters. 
		
		
In semi-supervised machine learning (ML) techniques, expert feedback is leveraged, but the result is often not interpretable and is hard to be evaluated by an expert. There are large datasets where it is possible to apply such ML techniques thanks to the increasing computational power of computers, but it becomes even more difficult to interpret results for such large datasets. Time series can be represented visually and hence advantageous over other formats such as attribute-value pairs, so it is quite valuable to have interpretable results for time series. This is especially important for semi-supervised systems because the expert who contributes to the result might want to evaluate it as well. 

In this thesis, we focus on the interpretability and explainability of the semi-supervised clustering of time series. We employ the most query-efficient technique to date, Constraint-based Repeated Aggregation and Splitting for Time Series ($ \mathrm{COBRAS^{TS}} $), which has been developed by the DTAI research group at KU Leuven. $ \mathrm{COBRAS^{TS}} $ exploits expert knowledge by querying users' constraints between pairs of time series instances. $ \mathrm{COBRAS^{TS}} $ iteratively aggregates and splits super-instances to obtain a clustering of fine granularity. $ \mathrm{COBRAS^{DTW}} $ is one of the implementations of $ \mathrm{COBRAS^{TS}} $ that uses Dynamic Time Warping (DTW) to measure similarity between time series instances. Unlike Euclidean distance, which strictly abides by the time axis, DTW finds the best alignment between two time series instances by warping and computes their distance based on it. DTW itself has good explanation ability thanks to the fact that the warping process is recorded. We also select DTW Barycenter Averaging (DBA) as the averaging method on time series data as it can capture the most important characteristic of time series instances, such as trends, peaks, and troughs, etc. Throughout the thesis, we mainly focus on explaining the clustering result of $ \mathrm{COBRAS^{DTW}} $. 

We propose in this thesis two new ways of interpreting the clustering result of $ \mathrm{COBRAS^{TS}} $. The first one is to provide users an overview of the clustering by showing the main properties of the clusters such as DBA and our newly proposed warping deviations in the context of time series. In this way, we can help users build a general and intuitive understanding of the clustering result. The second one is proposed to answer more specific questions such as why two time series data are in the same cluster or the different clusters. We chain two sequences that users have interest in with the constraints obtained in the process of $ \mathrm{COBRAS^{TS}} $. In this way, users can look into the clustering result from a more detailed perspective. 

		 

	\par \textbf{Keywords: explainability, $ \mathrm{COBRAS^{TS}} $, DBA, DTW} 
	\end{abstract} 
	
	% A list of figures and tables is optional
	\listoffigures
	%\listoftables
	% If you only have a few figures and tables you can use the following instead
	
	%\listoffiguresandtables
	% The list of symbols is also optional.
	% This list must be created manually, e.g., as follows:
	\chapter{List of Abbreviations and Symbols}
	\section*{Abbreviations}
	\begin{flushleft}
		\renewcommand{\arraystretch}{1.1}
		\begin{tabularx}{\textwidth}{@{}p{20mm}X@{}}
			
			DTW       & Dynamic Time Warping     \\ 
			
			DBA       & DTW Barycenter Averaging \\    
			COBRA    & Constraint-based Repeated Aggregation \\
			COBRAS     & Constraint-based Repeated Aggregation and Splitting \\
			$ \mathrm{COBRAS^{TS} }$    & Constraint-based Repeated Aggregation and Splitting for Time Series\\
				COBRAS     & Constraint-based Repeated Aggregation and Splitting \\
			$ \mathrm{COBRAS^{DTW} }$    & Constraint-based Repeated Aggregation and Splitting for Time Series using DTW as distance measure\\
		\end{tabularx}
	\end{flushleft}
	\iffalse
	\section*{Symbols}
	\begin{flushleft}
		\renewcommand{\arraystretch}{1.1}
		\begin{tabularx}{\textwidth}{@{}p{12mm}X@{}}
			42    & ``The Answer to the Ultimate Question of Life, the Universe,
			and Everything'' according to\\
			$c$   & Speed of light \\
			
		\end{tabularx}
	\end{flushleft}
	\fi
	
	% Now comes to the main text
	\mainmatter
	
	\chapter{Introduction}
\label{cha:intro}
The goal of semi-supervised learning is to combine the opinions of experts with existing information. COBRAS \cite{van2018cobras} the most query-efficient semi-supervised clustering method. It has a time series-based version, $ \mathrm{COBRAS^{TS}} $ \cite{van2018cobrasts}.  Clustering time series data can be of significant value in several scenarios in our daily life, such as clustering on the energy consumption data, electrocardiogram data, etc. However, how the clustering result is obtained is a black box or at least a gray box to end-users. The resulting clustering is difficult to interpret and evaluate, especially for large datasets. This is a significant drawback for semi-supervised systems where experts can contribute to the result but are unable to interpret or evaluate it. We study here how the semi-supervised clustering result can be explained and interpreted. Two benefits are doing so. Apart from providing end-users a more intuitive way of understanding the result, one can convince the experts who give opinions during the clustering process that the result is truly partly based on their knowledge.

\textbf{DTW.} DTW \cite{senin2008dynamic} is a method for computing the alignment and distance between time series instances. Unlike Euclidean distance, DTW computes the best possible alignment between two time series of respective lengths $  n $ and  $  m $  by first computing the $ n * m $ pairwise distance matrix between these points and then solving a dynamic program. The most important characteristic of DTW is its invariance against warping in the time axis. It can capture alternations between leading and lagging relationships of time series.  In other words, if two time series have similar trends but do not correspond to each other in the time axis, DTW will do a warping, making sure that similar trends inside them are mapped to each other. 

\textbf{DBA.} DBA \cite{petitjean2011global} is an averaging method for time series data. It does not calculate the average by the Euclidean distance. Instead, it uses the DTW mentioned above. Compared to the arithmetic mean, DBA preserves the ability of DTW, identifying time shifts. 

\textbf{COBRA.} COBRA \cite{vancraenendonck2018cobra} is a constraint-based semi-supervised clustering method. It allows users to give feedback on the relationship of instances. By introducing the concept of \textbf{super-instance} which are a group of similar instances, and only execute queries on the super-instance, COBRA can substantially reduce the number of queries, which makes it query-efficient.

\textbf{COBRAS.} COBRAS \cite{van2018cobras} is designed to automatically form clusters of better granularity. It refines COBRA by giving chance to split super-instances after they are formed. COBRAS beats COBRA for its ability to converge to clustering with better-grained levels of granularity in a shorter time with fewer queries. 

\textbf{$ \mathbf{COBRAS^{TS}} $.} $ \mathrm{COBRAS^{TS}} $ \cite{van2018cobrasts} is COBRAS with respect to time series.

\textbf{$ \mathbf{COBRAS^{DTW}} $.} $ \mathrm{COBRAS^{DTW}} $ is one of the implementations of the $ \mathrm{COBRAS^{TS}} $. As it uses DTW as its distance measure and DTW itself has good explainabilty, throughout the theis we explain the clustering result of $ \mathrm{COBRAS^{DTW}} $. When we refer $ \mathrm{COBRAS^{TS}} $ in this thesis, we are refering to the implementation $ \mathrm{COBRAS^{DTW}} $.

\textbf{Our contributions.} In this thesis, we propose two new ways to explain the time series clustering. 

Our first idea is to provide an overview of the clustering result by showing the properties of time series instances inside one cluster. We mainly show their DBA and deviation. Considering the instances are time series instances, we propose new types of deviation that can capture the alignment information of the time series data. The idea is new and it helps users build insight into the fact that DTW sacrifices the similarities of time series instances on the X-axis (time-axis) to win similarities of time series instances on the Y-axis (value-axis).

We not only illustrate the clustering result of $ \mathrm{COBRAS^{TS}} $ from a general perspective but also from a specific perspective. Our second idea is to show the relationship chain between two specific instances with constraint information provided by users to explain why they have been clustered together or apart. Moreover, the way we chain two time series instances with constraints is not designed specifically for time series data. It can also be applied in other variants of COBRAS as long as the instances are visually friendly. The idea of chaining two instances with the constraints can be also used in explaining other constraint-based semi-supervised learning.

\textbf{Structure}. After providing background material in chapter \ref{cha:related_work} and problem definition in chapter \ref{cha:problem_definition}, We introduce in chapter \ref{cha:properties_of_a_cluster} different types of deviation for time series instances in one cluster. We follow in chapter \ref{cha:cluster_membership}  illustrating the relationship chain between two time series instances with the constraints obtained in the process of $ \mathrm{COBRAS^{TS}}$. We close this paper in chapter \ref{cha:conclusion} with the pros and cons of my ways of explaining time-series clusters and show potential applications.
%%% Local Variables: 
%%% mode: latex
%%% TeX-master: "thesis"
%%% End: 
 
	
\chapter{Background and related work}
\label{cha:related_work}
Before all, time series related terminologies are introduced in more details.
These include distance measurement methods, averaging methods and semi supervised clustering methods in perspective of time series. Apart from the basic ones, I also introduce the variant ones.

\section{Dissimilarity measurement }
Euclidean distance is commonly used in measuring distances between two points, two vectors, etc. It is simple and competitive for many senarios. However, it is not the best choice in some domains, such as time series. Two time series might be sensitive to time axis, even thought they share similar shapes they might not match well in Euclidean distance. A new way of calculating distance between time series is thus proposed, which is DTW(Dynamic Time Wrapping)
\subsection{DTW}
Dynamic time warping (DTW) is a well-known method to measure the distance between time series instances. Unlike Euclidean distance, which is the sum of the distance of instances in each time axis, as shown in \ref{algo:dtw}, DTW finds an optimal alignment between two time series. Intuitively, it uses dynamic programming to map two time series in a way that their peaks and their bottoms are mapped together. In other words, DTW maps their trends instead of their time axis.

\begin{algorithm}[h] \label{algo:dtw}
	\caption{DTW (s: array[0...n], t: array[0...m])}
	\begin{algorithmic}[1]
		\STATE DTW := array[0..n, 0..m]
		\FOR {i := 0 to n}
		\FOR {j := 0 to m}
			\STATE DTW[i, j] := d(s[i], t[j])
		\ENDFOR
		\ENDFOR
		\FOR { i := 1 to n}
		 \STATE DTW[i, 0] += DTW[i-1, 0]
		\ENDFOR
		
		\FOR { j := 1 to n}
		\STATE DTW[0, j] += DTW[0, j-1]
		\ENDFOR
		
		\FOR { i := 1 to n}
		\FOR {j := 1 to m}	
		\STATE DTW[i, j] += min(DTW[i-1, j],									
							DTW[i, j-1],
							DTW[i-1, j-1])
		\ENDFOR	
		\ENDFOR
	   \STATE return DTW[n-1, m-1]						
		
	\end{algorithmic}
\end{algorithm}

\begin{figure}[htbp]
	\centering
	\begin{minipage}[t]{0.48\textwidth}
		\centering
		\includegraphics[width=6cm]{pics/related_work/dtw}
		\caption{DTW}
		\label{fig:dtw}
	\end{minipage}

\end{figure}
\subsection{others (eg. soft, k-shape, in brief)}
\subsection{motivation for selecting DTW}
DTW not only captures amplitude information of time series and computes their distance based on it, but also records how two time series are mapped to each other in the process of wrapping. In other words, the wrapping can be seen as evidence why two time series are similar to each other, or vise versa. DTW itself has good explanation ability.


\section{Averaging methods}
Apart from selecting the proper distance measure method for time series instances, choosing the right averaging method for time series instances is also of much importance. A suitable average of time series should reflect the overall trend of them as well as the peak and bottom message. 
\subsection{DBA}
DTW Barycenter Averaging(DBA) is designed as such that first initialize the average as the medoid of the time series instances, then update the average by averaging the wrapping of other time series instances from the current average for serveral times. Note that the wrapping of the time series instance from the current average is calculated with DTW. 
\subsection{others}
\subsection{motivation for using DBA}
As we can see from \fref{fig:arithmetic} and \fref{fig:dba}, DBA can capture the most important trend of instances, while the characteristic of the original instances are averaged in the time axis so the information about peaks and bottoms are blurred in the arithmetic mean. As a result, I select DBA as the averaging method in this paper. 

\begin{figure}[htbp]
	\centering
	\begin{minipage}[t]{0.48\textwidth}
		\centering
		\includegraphics[width=6cm]{pics/related_work/arithmetic_mean}
		\caption{Arithmetic mean}
		\label{fig:arithmetic}
	\end{minipage}
	\begin{minipage}[t]{0.48\textwidth}
		\centering
		\includegraphics[width=6cm]{pics/related_work/DBA}
		\caption{DBA}
		\label{fig:dba}
	\end{minipage}
\end{figure}

\section{Semi supervised time series clustering methods}

\subsection{COBRA}
 Clustering is inherently subjective. Different users might expect different clustering results given same dataset. 
 
 COBRA is a constraint-based semi-supervised clustering method. It first use k-means algorithm to split instances into several super instances. Each super instance contains serveral instances. Later, it exploits expert knowledge by asking users most informative quries and add contraints on super instances. These contraints include must-link, meaning two super instances should be clustered together, and cannot-link, meaning two super instances should not be in the same cluster. It is a bottom-up clustering. 
\subsection{COBRAS[detail]}
 There is one defect of COBRA. Once the super instance is formed, it will never change. COBRAS gives chance to super instance to split again later so that it can form clusters of better granuity.
\subsection{COBRAS-TS[detail]}
COBRAS-TS is the time series version of COBRAS. 
\subsection{others}
\subsection{reason of choice}



%%% Local Variables: 
%%% mode: latex
%%% TeX-master: "thesis"
%%% End: 

	\chapter{Problem definition}
\label{cha:problem_definition}

Many applications of machine learning involve time series data (e.g. sensor measurements). An important aspect of applying machine learning models to real-world use cases is interpretability. Good interpretability of models not only helps users comprehend them better and more easily but also motivates users to build insight into the models. In this thesis, we are interested in explaining the clustering of $\mathrm{COBRAS^{TS}}  $. We want to explore methods to summarize why a set of time series are clustered together and how one cluster differs from other clusters. To be more specific, our targets are shown below:

\begin{itemize}
	\item To show interpretable properties for each cluster in the context of time series (see \cref{cha:properties_of_a_cluster})
	\item To obtain an interpretable comparison between different clusters (see \cref{cha:properties_of_a_cluster})
	\item To explain why given time series appear in the same cluster or fall into different clusters (see \cref{cha:cluster_membership})
\end{itemize}


%%% Local Variables: 
%%% mode: latex
%%% TeX-master: "thesis"
%%% End: 

	\chapter{Explaining properties of a cluster}
\label{cha:4}

\section{Introduce the DTW deviation}
\subsection{Mathematically}
\section{result of experiments}
\subsection{multiple datasets}
	\chapter{Explaining cluster membership [rethink about a better title]}
\label{cha:6}

\subsection{must-link vs cannot-link}
\subsection{relationship between super instances based on must + cannot links}
\subsubsection{must link}
\subsubsection{must link + cannot link}
\subsection{algorithm for readers to re-implement (pseudo code)}
\subsection{result of experiments [Cases on differnt datasets and different queries]}
	
	\chapter{Conclusion}
\label{cha:conclusion}

In this thesis we propose two new ways to explain the clustering result of $\mathrm{COBRAS^{TS}}$. The first one is to show the main properties of a cluster such as DBA and our newly proposed warping deviations. The second one is to chain two sequences with constraints obtained during the process of $\mathrm{COBRAS^{TS}}$ to explain why they belong to the same cluster or why they are in different clusters.

\section{Explaining properties of a cluster}
Given the clustering result, we first provide users a general overview of each cluster by showing them the main properties such as average and deviations of the cluster. We choose DBA as the average because it uses DTW as the distance measure while $\mathrm{COBRAS^{DTW}}$ also uses DTW as the distance measure. Besides, DTW itself has good explainability. We propose warping deviations, horizontal warping deviation, and vertical warping deviation respectively. They are the deviations that reflect how much is warping in DTW. We can see that horizontal warping deviation is much bigger than the vertical warping deviation, which is in line with the fact that DTW warps the sequences in the time axis to shorten their distances in the value axis.

\section{Explaining relation between two sequences}
Apart from illustrating clustering in a general way, we also illustrate clustering from a specific perspective. We provide users the chance to argue the relation between two sequences, such as what leads them to be in the same cluster, or what separates them apart. We chain two sequences with the constraints obtained during the process of $\mathrm{COBRAS^{TS}}$ to answer users' questions. It gives users a window to look back whether and how their expert knowledge is applied in $\mathrm{COBRAS^{TS}}$. The idea can not only be used in $\mathrm{COBRAS^{TS}}$, but can also be used in other constraint-based clusterings as long as the instances are visualization friendly.


\subsection{Future works}
Here we provide possible improvements on the methods we propose.

\begin{enumerate}
	\item When presenting the properties of a cluster, we show the sequences inside the cluster. It might lead to heavy overlapping if the number of sequences is too big. In this case, we can consider only showing the representatives of the super-instances in the cluster
	
	\item When chaining two sequences with constraints obtained during $\mathrm{COBRAS^{TS}}$, we can confine the selected constraints to the ones that are truly used in merging stages. 
	
	\item We can record the "age" of the constraint and make sure the links on the constraint chain are relatively young because young constraints are more likely to contribute to the current clustering.
	
	\item We can give users a chance to select the longest length of the constraint chain in case it is too long and not user-friendly.
	
	
\end{enumerate}



%%% Local Variables: 
%%% mode: latex
%%% TeX-master: "thesis"
%%% End: 

	
	% If you have appendices:
	%\appendixpage*          % if wanted
	%\appendix
	%\chapter{The First Appendix}
\label{app:comparison_between_cluster}
Appendices hold useful data which is not essential to understand the work
done in the master's thesis. An example is a (program) source.
An appendix can also have sections as well as figures and references\cite{h2g2}.

\section{More Lorem}
\lipsum[50]


%%% Local Variables: 
%%% mode: latex
%%% TeX-master: "thesis"
%%% End: 

	% ... and so on until
	%\include{app-n}
	
	\backmatter
	% The bibliography comes after the appendices.
	% You can replace the standard "abbrev" bibliography style with another one.
	\bibliographystyle{abbrv}
	\bibliography{references}
	
\end{document}

%%% Local Variables: 
%%% mode: latex
%%% TeX-master: t
%%% End: 
