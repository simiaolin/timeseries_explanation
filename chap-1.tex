\chapter{background and related work}
\label{cha:1}
A chapter is a logical unit. It normally starts with an introduction, which
you are reading now. The last topic of the chapter holds the conclusion.

\section{Time wrapping method}
First comes the introduction to this topic.

\subsection{DTW [in introduction briefly, in background theoretically]}
Please don't abuse enumerations: short enumerations shouldn't use
``\verb|itemize|'' or ``\texttt{enumerate}'' environments.
So \emph{never write}: 
\begin{quote}
  The Eiffel tower has three floors:
  \begin{itemize}
  \item the first one;
  \item the second one;
  \item the third one.
  \end{itemize}
\end{quote}
But write:
\begin{quote}
  The Eiffel tower has three floors: the first one, the second one, and the
  third one.
\end{quote}
\subsection{others (eg. soft, k-shape, in brief)}
\subsection{reason of choice}
\section{average method}
\subsection{DBA}
\subsection{others [what if too long, be consistent,  can change the variable names of the equations]}
\subsection{reason of choice}

\section{Semi supervised time series clustering methods}
\subsection{COBRAS-TS}
\subsection{others}
\subsection{reason of choice}



%%% Local Variables: 
%%% mode: latex
%%% TeX-master: "thesis"
%%% End: 
