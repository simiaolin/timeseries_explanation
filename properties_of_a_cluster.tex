\chapter{Explaining properties of a cluster}
\label{cha:properties_of_a_cluster}
Given a time series clustering obtained by $ \mathrm{COBRAS^{TS}} $, we are interested in interpretable properties for each cluster. In the context of time series, such properties could be the average, variance, peak sand troughs, etc. Average is the most important characteristic. As for it, we have introduced DTW Barycenter Average (DBA), a global averaging method that can retain the important trends of the time series. However, as for the variance, the traditional variance no longer works well because neither the distance measure throughout $ \mathrm{COBRAS^{TS}} $ nor DBA is based on the Euclidean distance. Inspired by the DBA, we propose warping variance and warping deviation that are tailored for time series. 

\section{warping deviation}
Before introducing the warping variance, let's first look back how the time series average is obtained in DBA. In \cref{fig:dba} there are three sequences, S, Q and Avg, where Avg is the DBA of S and Q. Specifically, coordinate $ \mathrm{Avg_9} $ is associated with coordinates $ \mathrm{S_5} $, $ \mathrm{S_6} $ and $ \mathrm{S_7} $ in S, and coordinate $ \mathrm{Q_10} $ in Q. Remember the Barycenter Algorithm in \cref{algo:barycenter}, we have:

\begin{center}
	$ \mathrm{Avg_9 = \frac{S_5 + S_6 + S_7 + Q_{10}}  {4}}  $
\end{center} 

Traditionally, given for each coordinate in the average, variance is calculated as below:


\begin{algorithm}[h] \label{algo:variance}
	\caption{variance}
	
		\begin{algorithmic}[0]
		\REQUIRE $ \mathrm{C = <C_{1}, ...,C_{T^{'}} >}$ the average sequence with length $ T^{'} $
		\ \REQUIRE $ \mathrm{S_1 = <S_{1_1}, ...,S_{1_T} >}$ the 1st sequence to average 
		
		.
		

		\REQUIRE$ \mathrm{S_n = <S_{n_1}, ...,S_{n_T} >}$ the nth sequence to average 
		\REQUIRE all the sequences to average have length $ T $
		\REQUIRE $ T =  T^{'}$
		\noindent \STATE \textbf{Let} $ variance $ be a sequence of length $ T $
		\STATE $ variance \leftarrow [0, ..., 0] $
		
		\FOR {i = 1 to T}
		\STATE $variance[i] \leftarrow  \frac{{(S_{1_i} - C_{i})}^2  + ... + {(S_{n_i} - C_{i})}^2 } {T}$
		\ENDFOR
		\RETURN  $ variance $
	\end{algorithmic}
\end{algorithm}


\begin{algorithm}[h] \label{algo:std_deviation}
	\caption{standard deviation}
	
	\begin{algorithmic}[0]
		\REQUIRE $ \mathrm{C = <C_{1}, ...,C_{T^{'}} >}$ the average sequence with length $ T^{'} $
		\ \REQUIRE $ \mathrm{S_1 = <S_{1_1}, ...,S_{1_T} >}$ the 1st sequence to average 
		
		.
		
		
		\REQUIRE$ \mathrm{S_n = <S_{n_1}, ...,S_{n_T} >}$ the nth sequence to average 
		\REQUIRE all the sequences to average have length $ T $
		\REQUIRE $ T =  T^{'}$
		\noindent \STATE \textbf{Let} $ std\_deviation $ be a sequence of length $ T $
		\STATE $ std\_deviation \leftarrow [0, ..., 0] $
		
		\FOR {i = 1 to T}
		\STATE $std\_deviation[i] \leftarrow  \sqrt {\frac{{(S_{1_i} - C_{i})}^2  + ... + {(S_{n_i} - C_{i})}^2 } {T}}$
		\ENDFOR
		\RETURN  $ std\_deviation $
	\end{algorithmic}
\end{algorithm}




\begin{figure}[htbp]
	\centering
	\includegraphics[width=\linewidth]{pics/properties/dba}
	\caption{Association of coordinates from sequences to a coordinate of average}
	\label{fig:dba}
\end{figure}

\subsection{Mathematically}
\section{result of experiments}
\subsection{multiple datasets}