\chapter{Introduction}
\label{cha:intro}
The goal of semi-supervised learning is to combine opinions of experts together with existed information. COBRAS is most query-efficient semi-supervised clustering method. It has a time series based version, COBRAS-TS.  Clustering on time series data can be of significant value in a number of scenarios in our daily life, such as clustering on the energy consumption data, electrocardiogram data, etc. However, how the clustering result is obtained is a black box or at least gray box to end users. I study here the explanation of the clustering result. There are two benefits doing so. Apart from providing end users a more intuitive way of understanding the result, I can convince the experts who give opinions during the clustering process that the result is truly partly based on their knowledge.

\textbf{DTW.} Unlike Euclidean distance, DTW computes the best possible alignment between two time series of respective length $  n $ and  $  m $  by first computing the $ n * m $ pairwise distance matrix between these points and then solving a dynamic program. The most important characteristic of DTW is its invariance against wrapping in the time axis. It can capture alternations between leading and lagging relationships of time series.  In other words, if two time series have similar trend but not correspond to each other in time axis, DTW will do a wrapping, making sure that similar trends inside them are mapped to each other. 

\textbf{DBA.} Averaging on time series data is different from averaging on common data. DBA do not calculate the average by the Euclidean distance. Instead, it use the DTW mentioned above. Compared to arithmetic mean, DBA preserves the ability of DTW, identifying time shifts. 

\textbf{COBRAS.} COBRAS is a constraint-based clustering method. It allows users to give feedback on the relationship of instances and forms clusters based with fine-grained levels of granularity on those feedback. 

\textbf{COBRAS-TS.} COBRAS-TS is COBRAS with respect to time series.

\textbf{My contributions.} I mainly provide two ways to explain the time series clustering. One is to show the properties of time series instances inside one cluster. The other is to show the relationship entailment between two specific instances with constraint information provided by users to explain why they are clustered together or apart.

\textbf{Structure}. After providing background material in chapter \ref{cha:related_work} and problem definition in chapter \ref{cha:problem_definition}, I introduce in chapter \ref{cha:properties_of_a_cluster} different types of deviation for time series instances in one cluster. I follow in chapter \ref{cha:cluster_membership}  illustrating the relationship entailment between two time series instances with the help of constraints provided by users.  I close this paper in chapter \ref{cha:conclusion} with pros and cons of my ways of explaining time series cluster and show potential applications.
%%% Local Variables: 
%%% mode: latex
%%% TeX-master: "thesis"
%%% End: 
