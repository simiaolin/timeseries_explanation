\chapter{Explaining cluster membership }
\label{cha:cluster_membership}
Clustering is inherently subjective. We have introduced in \cref{sec:COBRA} that COBRA is a semi-supervised clustering method. It allows users to designate relations between instances to exploit expert knowledge. When presenting the clustering result to them, we should be able to convince them that their designation has been well applied and retained. 

COBRAS is an enhanced variant of CORBA that allows splitting on super-instances after they are formed to get clustering of better granularity. $\mathrm{COBRAS^{TS}}$ is a time series variant of COBRAS. We are interested in explaining the cluster membership of time series instances in $\mathrm{COBRAS^{TS}}$ because time series data are easier to visualize than other formats such as attribute-value pairs.

Providing users the properties of clusters is a good way to help them build an overview of the clusters. However, if they are interested in more specific aspects, like why two sequences are clustered together, or why they are separated from each other, the overall properties can do little. For instance, in \cref{fig:mbs_ecg_circled}, looking at the sequences of the largest cluster of ECG dataset with a query budget of 100 in $\mathrm{COBRAS^{TS}}$, people would easily question why the blue sequences are in this cluster. There are obvious higher peaks with values around 3 at an index around 92 in the blue sequences. However, there are neither such kind high peaks in other sequences in the cluster nor in the DBA of this cluster. There is no such peak in the DBA because the number of blue sequences is not comparable to that of the other sequences, so the peak information in the blue sequences is lost in DBA because DBA is a weighted average. However, why blue sequences are clustered together with other sequences is still a mystery to us. That is the intention we propose in this section the membership of a sequence in one cluster. We mainly illustrate the membership of sequences with the help of must-links and cannot-links. 

\begin{figure}[htbp]	
	\centering
	\includegraphics[width=\linewidth]{pics/membership/ecg_LINKS/cluster0_circled}
	\caption{Why these blue sequences are in the same cluster as other sequences?}
	\label{fig:mbs_ecg_circled}
\end{figure}

\section{Instances in one cluster}

$\mathrm{COBRAS^{TS}}$ forms clusters mainly based on the constraints provided by users and the similarity between sequences. There are two important stages in it, the splitting stage and merging stage, respectively. $\mathrm{COBRAS^{TS}}$ conducts these two stages alternately.

In the splitting stage, $\mathrm{COBRAS^{TS}}$ find the biggest super-instances that have the largest number of instances and splits it. To split the super-instance, it first determines the splitting level $\mathit{k}$ and then splits the super-instance to $\mathit{k}$ mini super-instances. $\mathrm{COBRAS^{TS}}$ queries users about constraints between sequences to determine the splitting level. These obtained constraints are stored and could be reused. $\mathrm{COBRAS^{TS}}$ only queries users the relation between two sequences when the relation of them could not be found in the stored constraints pool to save the query number as there is a query budget. When the splitting level is determined, it splits the super-instance using Spectral Clustering based on the affinity matrix containing the DTW distances between all the sequences in the super-instance. 

In the merging stage, $\mathrm{COBRAS^{TS}}$ first finds all the cluster pairs that are possible to be merged. A cluster pair ($\mathcal{C}_1, \mathcal{C}_2$) is considered as a merging candidate if there is no cannot-link between super-instance $\mathit{S_1}$ and super-instance $\mathit{S_2}$ as long as $\mathit{S_1}$ is in $ \mathcal{C}_1 $ and $\mathit{S_2}$ is in $ \mathcal{C}_2 $. All the candidate cluster pairs are sorted by their distance and queried one by one. Once a must-link between a candidate cluster pair is obtained, the candidate cluster pair is merged and $\mathrm{COBRAS^{TS}}$ continues to find new candidate cluster pairs, sort by their distances, and do the constraint checks and merging again as long as the query budget is not used up. The merging stage stops when the query budget is used up or there is no more merging candidate cluster pair. Like the determining splitting level stage in the splitting stage, all the obtained must-links and cannot-links are stored in the constraints pool during the merging stage. 

When querying the relation between a cluster pair ($\mathcal{C}_1, \mathcal{C}_2$), $\mathrm{COBRAS^{TS}}$ actually queries the relation between super-instances $\mathit{S_1}$ and $\mathit{S_2}$ where $\mathit{S_1}$ is in $\mathcal{C}_1$ and $\mathit{S_2}$ is in $\mathcal{C}_2$.
When querying the relation between super-instances $\mathit{S_1}$ and $\mathit{S_2}$, $\mathrm{COBRAS^{TS}}$ actually queries the relation between the representative sequence $\mathit{I_1}$ of super-instance $\mathit{S_1}$ and representative sequence $\mathit{I_2}$ of super-instance $\mathit{S_2}$. The representative sequence $\mathit{I}$ of a super-instance $\mathit{S}$ is the sequence in $\mathit{S}$ which has the smallest sum of distances to all the other sequences in $\mathit{S}$. When we talk about the relation between two super-instances, we are also talking about the relation between the representatives of these two super-instances, vice versa.

For every two super-instances $\mathit{S_1}$, $\mathit{S_2}$ in a cluster, either there is a direct must-link between them, or there are other super-instances $\mathit{S_3}$, $\mathit{S_4}$, ..., $\mathit{S_n}$ where there is a must-link between $\mathit{S_1}$ and $\mathit{S_3}$, a must-link between $\mathit{S_4}$ and $\mathit{S_5}$, ..., a must-link between $\mathit{S_{n-1}}$ and $\mathit{S_n}$, a must-link between $\mathit{S_n}$ and $\mathit{S_2}$. In other words, if there is no direct must-link between two super-instances in the same cluster, it is the constraint entailment that brings them together. Those must-links could be found in the constraint pool which is recorded during the process of $\mathrm{COBRAS^{TS}}$. With those must-links, every pair of super-instances in a cluster could be connected. Another fact is that each sequence belongs to a super-instance, and the instances inside every super-instance have high similarities to each other. Based on the two facts above, we can connect two sequences in a cluster. We illustrate it together with an example in \cref{fig:entailment}.

In \cref{fig:entailment}, we have one cluster $ \mathcal{C}_1 $. In $ \mathcal{C}_1 $, we have three super-instances, $\mathit{S_1}$, $\mathit{S_2}$, and $\mathit{S_3}$ respectively. In $\mathit{S_1}$, there are four sequences, and the representative of $\mathit{S_1}$ is $\mathit{I_1}$. In $\mathit{S_2}$, there are three sequences, and the representative of $\mathit{S_2}$ is $\mathit{I_2}$. In $\mathit{S_3}$, there are five sequences, and the representative of $\mathit{S_3}$ is $\mathit{I_3}$. We are interested in why sequence $\mathit{a}$ and $\mathit{b}$ is in the same cluster $ \mathcal{C}_1 $.

Firstly, connect each sequence to the representative of the super-instance it belongs to. If the sequence itself is already representative of the super-instance, skip this step. For instance in \cref{fig:entailment}, we would first find the representatives of the super-instances that $\mathit{a}$ and $\mathit{b}$ belong to. Sequence $\mathit{a}$ belongs to super-instance $\mathit{S_1}$, and the representative of $\mathit{S_1}$ is $\mathit{I_1}$, so we connect $\mathit{a}$ to $\mathit{I_1}$. Similarly, we also connect $\mathit{b}$ to $\mathit{I_3}$

Secondly, connect the representatives with the must-links in the constraint pool. We conduct the breadth-first search (BFS) among the must-links to find the must-link chain between two representatives. In this example, we are trying to connect $\mathit{I_1}$ and $\mathit{I_3}$. We can see that there is a sequence $\mathit{I_2}$ which is the representative of super-instance $\mathit{S_2}$, and $\mathit{S_2}$ is also in the cluster $ \mathcal{C}_1 $. Besides, there is a must-link between $\mathit{I_1}$ and $\mathit{I_2}$, and there is also a must-link between $\mathit{I_2}$ and $\mathit{I_3}$. Therefore, we can connect $\mathit{I_1}$ and $\mathit{I_3}$ by first connecting $\mathit{I_1}$ and $\mathit{I_2}$ and then connecting $\mathit{I_2}$ and $\mathit{I_3}$. 

Finally, combine all the must-links selected above, we can connect $\mathit{a}$ and $\mathit{b}$ in a must-link chain $\mathit{a-I_1-I_2-I_3-b}$, which is shown as green dotted lines in \cref{fig:entailment}. 

\begin{figure}[htbp]	
	\centering
	\includegraphics[width=\linewidth]{pics/membership/entailment}
	\caption{Why two sequences are in one cluster?}
	\label{fig:entailment}
\end{figure}

\section{Instances in two clusters}
As for two sequences in different clusters, we can slightly change the way we find the must-link chain above to explain why they are in two clusters. As long as the query budget is adequate, there is only one reason that two clustered are not merged together - there exists at least one cannot-link between these two clusters. Combining this cannot-link together with the must-links inside two clusters, we can get a constraint chain that can explain why two sequences are separated into two clusters. We illustrate it together with an example in \cref{fig:transitivity}.

In \cref{fig:transitivity} there are two clusters, $ \mathcal{C}_1 $ and $ \mathcal{C}_2 $. $ \mathcal{C}_1 $ is just the same cluster in \cref{fig:entailment}, $ \mathcal{C}_2 $ contains two super-instances, $\mathit{S_4}$ and $\mathit{S_5}$ repectively. $\mathit{I_4}$ is the representative of $\mathit{S_4}$, and $\mathit{I_5}$ is the representative of $\mathit{S_5}$. $\mathit{c} $ is a sequence in $\mathit{S_5}$. We are interested in why sequence $\mathit{a}$ and $\mathit{c}$ are not in the same cluster.

Firstly, find which two clusters the two target sequences are grouped into, then find a cannot-link between two clusters. Specifically, find a cannot-link between two super-instances that belong to two clusters respectively. More specifically, find a cannot-link between two representatives of two super-instances that belong to two clusters respectively. There might exist more than one such cannot-link, we just randomly pick one. In this case, a cannot-link between $\mathit{I_3}$ and $\mathit{I_4}$ satisfies our requirement as $\mathit{I_3}$ is the representative of super-instance $\mathit{S_3}$ in cluster $ \mathcal{C}_1 $, while $\mathit{I_4}$ is the representative of super-instance $\mathit{S_4}$ in cluster  $ \mathcal{C}_2 $. 

Secondly, connect each sequence to the representative of the super-instance it belongs to. If the sequence itself is already representative of the super-instance, skip this step. In this case, we connect $\mathit{a}$ to $\mathit{I_1}$, and connect $\mathit{c}$ to $\mathit{I_5}$.

Thirdly, for the representatives found in step one and step two that are in the same cluster, conduct BFS in the constraint pool and connect them with the must-link chain. For instance, we find $\mathit{I_3}$ in $ \mathcal{C}_1 $ in step one and $\mathit{I_1}$ in $ \mathcal{C}_1 $ in step two, so we connect them using must-link chain $\mathit{I_1-I_2-I_3}$. Similarly, as we find $\mathit{I_4}$ in $ \mathcal{C}_2 $ in step one and $\mathit{I_5}$ in $ \mathcal{C}_2 $ in step two, we connect them using must-link chain $\mathit{I_4-I_5}$.

Finally, combine all the must-links and cannot-link found above, we have constraint chain $\mathit{a-I_1-I_2-I_3-I_4-I_5-c}$, where $\mathit{I_3-I_4}$ shown in red dotted line is a cannot-link, and all the other constraints shown in green dotted lines are must-links.


\begin{figure}[htbp]	
	\centering
	\includegraphics[width=\linewidth]{pics/membership/transitivity}
	\caption{Why two sequences are in different clusters?}
	\label{fig:transitivity}
\end{figure}

\section{Experiments}
We design an interactive system that allows users to designate the sequence ids of two sequences in which they have an interest in their relation. Every time they input two indices, we plot the constraint chain for them. There is a must-link between sequences with the same color except when they are two sequences at the beginning or at the end of the constraint chain. In that case, there might not exist a must-link between them. Instead, they are in the same super-instance and one of them is the representative of the super-instance. There is a cannot-link between sequences that are located on the border where the color changes. The datasets and configuration are the same as that of experiments in \cref{cha:properties_of_a_cluster}.


\subsection{Experiment on ECG dataset}
We first experimented with the ECG dataset, and we set the query budget of $ \mathrm{COBRAS^{TS}} $ to 100. 

We are interested in why the blue sequence with a high peak with a value around 3 at an index around 92 could be clustered together with other sequences that do not have such a peak, so we provide the system the id of the blue sequence and the id of a sequence in the same cluster without such a peak, which are 125 and 66 respectively. The result is shown in \cref{fig:member_ml_ecg}. From bottom to top are the sequences in the computed constraint chain. The constrain link is $\mathit{66-78-125}$. We can now clearly know that it is the sequence of id 78 that links the sequence of id 125 with a high peak at the tail and the sequence of id 66 together into the same cluster.

\begin{figure}[htbp]	
	\centering
	\includegraphics[width=\linewidth]{pics/membership/ecg_LINKS/ml_cluster0}
	\caption{Must-link chain in ECG}
	\label{fig:member_ml_ecg}
\end{figure}

We are also interested in why two sequences are not clustered together. For example, we want to explore why the sequence of id 84 and sequence of id 198 are in two clusters. We provide the ids to the system, and it shows us the result in \cref{fig:member_cl_ecg}. Here the sequences in red are in the same cluster, and sequences in green are in another cluster. The constraint link is $\mathit{84-63-143-125-198}$. There is a cannot-link between the red sequence of id 143 and the green sequence of id 125. The sequence of id 84 is linked to the sequence of id 143 via must-links $\mathit{84-63-143}$, while the sequence of id 198 is linked to sequence 125 via must-links $\mathit{198-125-143}$. 

\begin{figure}[htbp]	
	\centering
	\includegraphics[width=\linewidth]{pics/membership/ecg_LINKS/cl_1_and_0}
	\caption{Must-link and cannot-link chain in ECG}
	\label{fig:member_cl_ecg}
\end{figure}


\subsection{Experiment on Trace dataset}
We also experimented on the Trace dataset, and we set the query budget of $ \mathrm{COBRAS^{TS}} $ to 100. 

In \cref{fig:member_ml_trace}, we can see that the sequence of id 55 is in the same cluster with the sequence of id 115 because of the must-links $\mathit{55-173-115}$. 

In \cref{fig:member_cl_trace}, we can see that seqeucne of id 143 and sequece of id 115 are in two different clusters because of the constraint chain $\mathit{143-95-10-174-135-88-115}$, where $\mathit{174-135}$ is a cannot-link.

\begin{figure}[htbp]	
	\centering
	\includegraphics[width=\linewidth]{pics/membership/trace_links/must}
	\caption{Must-link chain in Trace}
	\label{fig:member_ml_trace}
\end{figure}

\begin{figure}[htbp]	
	\centering
	\includegraphics[width=\linewidth]{pics/membership/trace_links/cannot}
	\caption{Must-link and cannot-link chain in Trace}
	\label{fig:member_cl_trace}
\end{figure}

\section{Conclusion}
In this chapter, we propose a new way to solve the doubt in the last chapter that why two sequences are in the same cluster, or why they are in different clusters. We solve in a way making good use of the obtained constraint in the process the $\mathrm{COBRAS^{TS}}$. Our system provides users the chance to look at the clustering result of $\mathrm{COBRAS^{TS}}$ in a more specific perspective. More importantly, we can convince users that their expert knowledge is truly and well obtained and exploited in $\mathrm{COBRAS^{TS}}$. 

