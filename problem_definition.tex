\chapter{Problem definition}
\label{cha:problem_definition}

Many applications of machine learning involve time series data (e.g. sensor measurements). An important aspect of applying machine learning models to real-world use cases is interpretability. Good interpretability of models not only helps users comprehend them better and more easily but also motivates users to build insight into the models. In this thesis, we are interested in explaining the clustering of $\mathrm{COBRAS^{TS}}  $. We want to explore methods to summarize why a set of time series are clustered together and how one cluster differs from other clusters. To be more specific, our targets are shown below:

\begin{itemize}
	\item To show interpretable properties for each cluster in the context of time series (see \cref{cha:properties_of_a_cluster})
	\item To obtain an interpretable comparison between different clusters (see \cref{cha:properties_of_a_cluster})
	\item To explain why given time series appear in the same cluster or fall into different clusters (see \cref{cha:cluster_membership})
\end{itemize}


%%% Local Variables: 
%%% mode: latex
%%% TeX-master: "thesis"
%%% End: 
